%%%%%%%%%%%%%%%%%%%%%%%%%%%%%%%%%%%%%%%%%%%%%%%%%%%%%%%%%%%%%%%%%%
%%%%%%%% ICML 2013 EXAMPLE LATEX SUBMISSION FILE %%%%%%%%%%%%%%%%%
%%%%%%%%%%%%%%%%%%%%%%%%%%%%%%%%%%%%%%%%%%%%%%%%%%%%%%%%%%%%%%%%%%

% Use the following line _only_ if you're still using LaTeX 2.09.
%\documentstyle[icml2013,epsf,natbib]{article}
% If you rely on Latex2e packages, like most moden people use this:
\documentclass{article}

% For figures
\usepackage{graphicx} % more modern
%\usepackage{epsfig} % less modern
\usepackage{subfigure} 

% For citations
\usepackage{natbib}

% For algorithms
\usepackage{algorithm}
\usepackage{algorithmic}

% Alkis
\usepackage{amsmath,amssymb}
\usepackage{bm}
\usepackage{amsthm}
\usepackage{xspace}          % Controls space after user-defined command
\usepackage{fixltx2e}        % For subscripts in normal text
%

% As of 2011, we use the hyperref package to produce hyperlinks in the
% resulting PDF.  If this breaks your system, please commend out the
% following usepackage line and replace \usepackage{icml2013} with
% \usepackage[nohyperref]{icml2013} above.
\usepackage{hyperref}

% Packages hyperref and algorithmic misbehave sometimes.  We can fix
% this with the following command.
\newcommand{\theHalgorithm}{\arabic{algorithm}}

% Employ the following version of the ``usepackage'' statement for
% submitting the draft version of the paper for review.  This will set
% the note in the first column to ``Under review.  Do not distribute.''
%\usepackage{icml2013} 
% Employ this version of the ``usepackage'' statement after the paper has
% been accepted, when creating the final version.  This will set the
% note in the first column to ``Proceedings of the...''
\usepackage[accepted]{icml2013}

% Alkis
\newcommand{\todo}[1]{\noindent\texttt{\color[rgb]{0.5,0.1,0.1} TODO: #1}}

\def\*#1{\bm{#1}}

\newcommand*\LNot{\textbf{not}\xspace}
\newcommand*\LAnd{\textbf{and}\xspace}
\newcommand*\LET[2]{\STATE #1 $\gets$ #2}
\newcommand*\Fcall[1]{\textsc{#1}}

\newcommand{\sectref}[1]{\hyperref[#1]{\mbox{Section~\ref*{#1}}}}
\newcommand{\figref}[1]{\hyperref[#1]{\mbox{Figure~\ref*{#1}}}}
\newcommand{\tabref}[1]{\hyperref[#1]{\mbox{Table~\ref*{#1}}}}
\newcommand{\algoref}[1]{\hyperref[#1]{\mbox{Algorithm~\ref*{#1}}}}
\newcommand{\theoremref}[1]{\hyperref[#1]{\mbox{Theorem~\ref*{#1}}}}
\newcommand{\lemmaref}[1]{\hyperref[#1]{\mbox{Lemma~\ref*{#1}}}}
\newcommand{\corref}[1]{\hyperref[#1]{\mbox{Corollary~\ref*{#1}}}}
\newcommand{\eqtref}[1]{\hyperref[#1]{\mbox{(\ref*{#1})}}}
\newcommand{\appref}[1]{\hyperref[#1]{\mbox{Appendix~\ref*{#1}}}}

\newcommand{\argmax}{\operatornamewithlimits{argmax}}
\newcommand{\argmin}{\operatornamewithlimits{argmin}}

\newcommand{\twopartdef}[4]
{
	\left\{
		\begin{array}{ll}
			#1\,,& \mbox{if } #2 \\
			#3\,,& \mbox{if } #4
		\end{array}
	\right.
}

\newtheorem{theorem}{Theorem}
\newtheorem{lemma}{Lemma}
\newtheorem{cor}{Corollary}

\renewcommand{\algorithmicrequire}{\textbf{Input:}}
\renewcommand{\algorithmicensure}{\textbf{Output:}}

\newcommand{\acl}{\textsf{ACL}\xspace}
\newcommand{\bacl}{\textsf{ACL\textsubscript{batch}}\xspace}
\newcommand{\fb}{\mathop{\mathrm{fb}}}
%

% The \icmltitle you define below is probably too long as a header.
% Therefore, a short form for the running title is supplied here:
\icmltitlerunning{Notes}

\begin{document} 

\twocolumn[
\icmltitle{Notes}

% It is OKAY to include author information, even for blind
% submissions: the style file will automatically remove it for you
% unless you've provided the [accepted] option to the icml2013
% package.
\icmlauthor{Alkis Gkotovos}{alkisg@student.ethz.ch}
\icmladdress{Computer Science Department, ETH Zurich}

% You may provide any keywords that you 
% find helpful for describing your paper; these are used to populate 
% the "keywords" metadata in the PDF but will not be shown in the document
\icmlkeywords{}

\vskip 0.3in
]

\begin{abstract}\end{abstract} 

\section{Gaussian Processes}
Given a vector of points $\*x_A = \left(\*x_v\right)_{v\in A}$ in input space
and their respective measurements
$\*y_A = \left(y_v\right)_{v\in A}$, where $y_v = f(\*x_v) + \epsilon$
with $\mathbb{E}[\epsilon] = 0$ and $\mathbb{V}[\epsilon] = \sigma_n^2$,
the predictive posterior distribution of any vector $\*x_B$ in input
space is a Gaussian
\begin{align*}
f(\*x_B)\mid \*y_A \sim \mathcal{N}\left(\mu_{B\mid A}, \Sigma_{B\mid A}\right)
\end{align*}
with
\begin{align*}
\mu_{B\mid A} &= K_{BA}\left(K_{AA} + \sigma_n^2 I\right)^{-1}\*y_A\\
\Sigma_{B\mid A} &= K_{BB} - K_{BA}\left(K_{AA} + \sigma_n^2 I\right)^{-1}K_{AB}.
\end{align*}
If $A = \left\{a_1, a_2, \ldots, a_{|A|}\right\}$, $B = \left\{b_1, b_2, \ldots, b_{|B|}\right\}$,
and we define $k_{ij} = k(\*x_{a_i}, \*x_{b_j})$,
then $K_{AB} = \left[k_{ij}\right]$ of size $|A|\times |B|$. Matrices $K_{AA}$, $K_{BA}$, and
$K_{BB}$ are defined analogously.

\section{Conditional Mutual Information}
Note that
\begin{align*}
\*y_B\mid \*y_A \sim \mathcal{N}\left(\mu_{B\mid A}, \Sigma_{B\mid A} + \sigma_n^2 I\right)
\end{align*}
and
\begin{align*}
\*y_B\mid \*f_B \sim \mathcal{N}\left(0, \sigma_n^2 I\right).
\end{align*}
Then, the conditional mutual information between measurements $\*y_B$ and $f$,
given measurements $\*y_A$ can be expressed as
\begin{align*}
&I(\*y_B; f \mid \*y_A)\\
&=H(\*y_B\mid \*y_A) - H(\*y_B\mid \*y_A, f)\\
&=H(\*y_B\mid \*y_A) - H(\*y_B\mid \*f_B)\\
&=\frac{1}{2}\log\left\{(2\pi e)^{|B|}\left|\Sigma_{B\mid A} + \sigma_n^2 I\right|\right\} -
  \frac{1}{2}\log\left\{(2\pi e)^{|B|}\left|\sigma_n^2 I\right|\right\}\\
&=\frac{1}{2}\log\left\{\left|I + \sigma_n^{-2}\Sigma_{B\mid A}\right|\right\}.
\end{align*}

\end{document} 


% This document was modified from the file originally made available by
% Pat Langley and Andrea Danyluk for ICML-2K. This version was
% created by Lise Getoor and Tobias Scheffer, it was slightly modified  
% from the 2010 version by Thorsten Joachims & Johannes Fuernkranz, 
% slightly modified from the 2009 version by Kiri Wagstaff and 
% Sam Roweis's 2008 version, which is slightly modified from 
% Prasad Tadepalli's 2007 version which is a lightly 
% changed version of the previous year's version by Andrew Moore, 
% which was in turn edited from those of Kristian Kersting and 
% Codrina Lauth. Alex Smola contributed to the algorithmic style files.  
